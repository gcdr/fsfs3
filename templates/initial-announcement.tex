\chapter{The Initial Announcement of the GNU~Operating~System}
\label{Initial Announcement}
\fcn{Copyright \copyright{} 1983 Richard Stallman }{\indent{This announcement is part of \fsfsthreecite}}

This is the original announcement of the GNU Project, posted by
Richard Stallman on 27~September~1983.

The actual history of the GNU Project differs in many ways from
this initial plan. For example, the beginning was delayed until
January 1984. Several of the philosophical concepts of free software
were not clarified until a few years later.

%%% TODOFSFS @rgindex Stallman, Richard
%%% TODOFSFS @rgindex GNU, initial announcement
%%% TODOFSFS @rgindex MIT (Massachusetts Institute of Technology)
%%% TODOFSFS @cindex Unix compatibility, announcement of
%%% TODOFSFS @cindex RMS, Richard Matthew Stallman, often referred to as \emph{(see also} Stallman\emph{)}
%%% TODOFSFS @cindex GNU, operating system parts
%%% TODOFSFS @cindex Empire (game)
%%% TODOFSFS @cindex games, Empire
%%% TODOFSFS @cindex C programs
%%% TODOFSFS @cindex C compiler
%%% TODOFSFS @cindex YACC (Yet Another Compiler Compiler)
\begin{quot}
From mit-vax!mit-eddie!RMS@MIT-OZ\\
From: \code{RMS\%MIT-OZ@mit-eddie}\\
Newsgroups: net.unix-wizards,net.usoft\\
Subject: new Unix implementation\\
Date: Tue, 27-Sep-83 12:35:59 EST\\
Organization: MIT AI Lab, Cambridge, MA

\medskip

Free Unix!

\medskip

Starting this Thanksgiving I am going to write a complete
Unix-compatible software system called GNU (for Gnu's Not Unix), and
give it away free\footnote{
The wording here was careless. The intention was that nobody would
have to pay for \emph{permission} to use the GNU system. But the words
don't make this clear, and people often interpret them as saying that
copies of GNU should always be distributed at little or no
charge. That was never the intent.
% @vskip -1.5pc
} to everyone who can use it.  Contributions of time, money, programs
and equipment are greatly needed.

\medskip

To begin with, GNU will be a kernel plus all the utilities needed to
write and run C programs: editor, shell, C compiler, linker,
assembler, and a few other things. After this we will add a text
formatter, a YACC, an Empire game, a spreadsheet, and hundreds of
other things. We hope to supply, eventually, everything useful that
normally comes with a Unix system, and anything else useful, including
on-line and hardcopy documentation.

\medskip

%%% TODOFSFS @cindex Chaosnet \emph{(see also} MIT\emph{)}
%%% TODOFSFS @cindex MIT, Chaosnet
%%% TODOFSFS @cindex UUCP (Unix-to-Unix Copy)
%%% TODOFSFS @cindex Lisp-based window system and programs
%%% TODOFSFS @cindex MIT, Chaosnet
GNU will be able to run Unix programs, but will not be identical to
Unix. We will make all improvements that are convenient, based on our
experience with other operating systems. In particular, we plan to
have longer filenames, file version numbers, a crashproof file system,
filename completion perhaps, terminal-independent display support, and
eventually a Lisp-based window system through which several Lisp
programs and ordinary Unix programs can share a screen.  Both C and
Lisp will be available as system programming languages.  We will have
network software based on MIT's chaosnet protocol, far superior to
UUCP. We may also have something compatible with UUCP.

\medskip

Who Am I?

\medskip

%%% TODOFSFS @cindex Emacs, GNU
%%% TODOFSFS @cindex MIT, AI (Artificial Intelligence) Lab
%%% TODOFSFS @cindex ITS (Incompatible Timesharing System)
%%% TODOFSFS @cindex Lisp Machine operating system
I am Richard Stallman, inventor of the original much-imitated EMACS
editor, now at the Artificial Intelligence Lab at MIT. I have worked
extensively on compilers, editors, debuggers, command interpreters,
the Incompatible Timesharing System and the Lisp Machine operating
system.  I pioneered terminal-independent display support in ITS. In
addition I have implemented one crashproof file system and two window
systems for Lisp machines.
%%% TODOFSFS @rgindex Stallman, Richard

\medskip

Why I Must Write GNU

\medskip

%%% TODOFSFS @cindex Golden Rule
%%% TODOFSFS @cindex citizen values, Golden Rule
%%% TODOFSFS @cindex nondisclosure agreements
I consider that the golden rule requires that if I like a program I
must share it with other people who like it. I cannot in good
conscience sign a nondisclosure agreement or a software license
agreement.

\medskip

So that I can continue to use computers without violating my
principles, I have decided to put together a sufficient body of free
software so that I will be able to get along without any software that
is not free.

\medskip

How You Can Contribute

\medskip

%%% TODOFSFS @cindex development, contributions and donations
%%% TODOFSFS @cindex call to action, initial announcement
I am asking computer manufacturers for donations of machines and
money.  I'm asking individuals for donations of programs and work.

\medskip

One computer manufacturer has already offered to provide a
machine. But we could use more. One consequence you can expect if you
donate machines is that GNU will run on them at an early date. The
machine had better be able to operate in a residential area, and not
require sophisticated cooling or power.

\medskip

%%% TODOFSFS @cindex Unix compatibility, ease of contribution because of
Individual programmers can contribute by writing a compatible
duplicate of some Unix utility and giving it to me. For most projects,
such part-time distributed work would be very hard to coordinate; the
independently-written parts would not work together. But for the
particular task of replacing Unix, this problem is absent. Most
interface specifications are fixed by Unix compatibility. If each
contribution works with the rest of Unix, it will probably work with
the rest of GNU.

\medskip

If I get donations of money, I may be able to hire a few people full
or part time. The salary won't be high, but I'm looking for people for
whom knowing they are helping humanity is as important as money. I
view this as a way of enabling dedicated people to devote their full
energies to working on GNU by sparing them the need to make a living
in another way.

\medskip

For more information, contact me.\\
Arpanet mail:\\
  RMS@MIT-MC.ARPA

\medskip

Usenet:\\
  ...!mit-eddie!RMS@OZ \quad\quad ...!mit-vax!RMS@OZ

\medskip

US Snail:\\
  Richard Stallman\\
  166 Prospect St\\
  Cambridge, MA 02139
\end{quot}
%%% TODOFSFS @rgindex GNU, initial announcement
%%% TODOFSFS @rgindex MIT (Massachusetts Institute of Technology)
